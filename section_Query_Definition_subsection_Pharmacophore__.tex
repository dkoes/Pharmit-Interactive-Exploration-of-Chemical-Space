\section{Query Definition}

Table of Contents:
Getting Started
Initiating a Search
Visualization
Pharmacophore Representations
Shape Constraints
Hit Reduction and Feasibility Screening
Database Selection
Saving and Loading Sessions
Managing Search Results
Saving and Minimizing Results
User-submitted Databases
Related Software
Additional Help
Getting Started
pharmit is a web server that facilitates virtual screening: it enables users to search for small molecules based on their structural and chemical similarity to another small molecule, with the goal of identifying those that bind to a target molecule (typically a protein receptor or enzyme). The search can take as input either a small molecule, a set of pharmacophore features (which will be subsequently described in greater detail), or both a protein and a small molecule for which a putative binding pose is known. These may be obtained from the Protein Data Bank (PDB) or uploaded by the user directly, using one of the two options provided on the main page.


Getting started
Initiating a Search
A structure can be obtained directly from the PDB by entering its four-character PDB identifier in the first text box after "start from PDB". A list of possible small molecules will be generated automatically in the second box, from which the user may choose a ligand of interest. Binding site waters may be excluded entirely, used as part of the ligand as optional pharmacophore features to include in the similarity search, or used as part of the receptor to identify which pharmacophore features of the ligand are relevant to binding. To proceed with the search, the user should next choose "submit". 
Alternatively the user may first choose "enter pharmit search" in order to upload structural files. After being redirected to the search page, choose "Load Features..." to upload a small molecule structure in sdf, pdb, mol2, or xyz format or a pharmacophore query file in MOE, LigBuilder, LigandScout, or Pharmer format. If a receptor is provided, it should be one for which the binding pose of the provided ligand is known, or to which the ligand was previously docked. pharmit will not dock the two compounds, and since the receptor is used to identify which pharmacophore features of the ligand are relevant to binding, providing a pair of structures that are oriented arbitrarily will fail to identify a relevant pharmacophore. The receptor may be provided in sdf, pdb, mol2, or xyz format.


Loading structures
Visualization
After the desired structures are provided, pharmit will identify all pharmacophore features present in the ligand if a ligand structure rather than a pharmacophore query file was provided. If a receptor structure was provided, it will identify which of these features are relevant to the interaction between the protein-ligand pair using distance cutoffs between interacting features and will display only these interacting features. Next, it will center the features and use a set of default visualization options to display the provided structures, including the electrostatic surface of the receptor. If the user is dissatisfied with the default visualization scheme, they may scroll down to the "Visualization" menu in the left sidebar and toggle the options as desired. In particular, if the electrostatic surface of the receptor is obscuring the pharmacophore features, receptor surface opacity may be reduced and at the lowest setting becomes entirely transparent. In the graphical display window, the left mouse button may be used to rotate the scene and the right mouse button or center wheel may be used to zoom. Clicking on spheres - both pharmacophore spheres and shape constraint spheres - toggles between a solid and wire display. In order to maximize the viewable graphical display area, the sidebar may be hidden by pressing the left-pointing arrowhead at the top right of the sidebar.


\subsection{Pharmacophore Queries}
A pharmacophore \cite{Koes_2015rev,Yang_2010,Leach_2010} defines the essential features of an interaction, such as hydrogen bond, charged, hydrophobic, or aromatic features. Importantly, a pharmacophore includes the spatial arrangement of these features and pharmacophore


Pharmacophores are used to define molecular similarity and perform structural alignment. A pharmacophore describes the spatial arrangement of the essential features of an interaction. Each pharmacophore query feature includes both a type and a radius. The pharmacophore type derives from the underlying chemistry of functional groups and in the context of pharmit may be defined as an aromatic, hydrogen donor, hydrogen acceptor, hydrophobic, negative ion, or positive ion. The set of pharmacophore features chosen to represent a particular ligand and their geometric orientation is used to identify other molecules in a database that also possess that configuration of features. The radius of a pharmacophore query feature determines how closely a molecule in the database must match the configuration of the query. To be considered similar, the pharmacophore feature points of a molecule in the database must overlap the corresponding pharmacophore feature spheres in the query. The likelihood of this occurring increases with larger feature radius.

A list of pharmacophore features is provided in the "Pharmacophore" menu in the sidebar. Pharmacophore features can be toggled "on" or "off" to include them in or remove them from the search. Pressing the "x" to the right of the pharmacophore removes it from the menu entirely. By pressing the arrow to the left of a given pharmacophore, the user can vary the type, radius, and location of a pharmacophore, as well as optionally restrict the number of atoms associated with a hydrophobic pharmacophore and the phi or psi angle associated with pharmacophore features, such as hydrogen bonds, that have an associated direction.


\subsection{Shape Queries}

\cite{matchpack}\cite{vams}
Users have the option of using shape as an additional constraint on the similarity search. The ligand's surface or a set of user-defined spheres may be used as an inclusive constraint, the receptor's surface or a set of user-defined spheres may be used as an exclusive constraint, or both inclusive and exclusive shapes may be used. Multiple spheres may be defined for both inclusive and exclusive search by choosing the "Spheres" option in the dropdown menu and pressing "Add". This will display a menu allowing the user to set the position and size of the sphere. Pressing "Add" again will enable the user to continue adding spheres. The shape constraint is applied differently depending on whether it is used to search the database prior to pharmacophore filtering or to filter results returned by a pharmacophore search. In the former case, a shape search returns molecules in the search database that contain the entire included shape and do not overlap any part of the excluded shape. In the latter case, the results of the pharmacophore search are filtered by shape to ensure that at least one heavy atom falls within the inclusive shape and no heavy atom centers fall within any exclusive shape. In either case, the results of the initial search are aligned to that search for the secondary filtering - thus for a given molecule a pose may exist that allows it to meet the secondary filter, but it will not be returned if the aligned pose resulting from the initial search does not meet those constraints. Our recommendation is that users define both inclusive and exclusive shape constraints using the ligand and receptor surfaces and use them as secondary filtering on the results of a pharmacophore search. The order of application of pharmacophore and shape constraints can be swapped using the second button from the top of the left sidebar, which switches between "Pharmacophore Search → Shape Filter" and "Shape Search → Pharmacophore Filter".
