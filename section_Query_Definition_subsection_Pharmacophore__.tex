\section{Query Definition}

\subsection{Pharmacophore Queries}
A pharmacophore \cite{Koes_2015rev,Yang_2010,Leach_2010} defines the essential features of an interaction, such as hydrogen bond, charged, hydrophobic, or aromatic features. Importantly, a pharmacophore includes the spatial arrangement of these features and pharmacophore


Pharmacophores are used to define molecular similarity and perform structural alignment. A pharmacophore describes the spatial arrangement of the essential features of an interaction. Each pharmacophore query feature includes both a type and a radius. The pharmacophore type derives from the underlying chemistry of functional groups and in the context of pharmit may be defined as an aromatic, hydrogen donor, hydrogen acceptor, hydrophobic, negative ion, or positive ion. The set of pharmacophore features chosen to represent a particular ligand and their geometric orientation is used to identify other molecules in a database that also possess that configuration of features. The radius of a pharmacophore query feature determines how closely a molecule in the database must match the configuration of the query. To be considered similar, the pharmacophore feature points of a molecule in the database must overlap the corresponding pharmacophore feature spheres in the query. The likelihood of this occurring increases with larger feature radius.

A list of pharmacophore features is provided in the "Pharmacophore" menu in the sidebar. Pharmacophore features can be toggled "on" or "off" to include them in or remove them from the search. Pressing the "x" to the right of the pharmacophore removes it from the menu entirely. By pressing the arrow to the left of a given pharmacophore, the user can vary the type, radius, and location of a pharmacophore, as well as optionally restrict the number of atoms associated with a hydrophobic pharmacophore and the phi or psi angle associated with pharmacophore features, such as hydrogen bonds, that have an associated direction.


\subsection{Shape Queries}

\cite{matchpack}\cite{vams}
Users have the option of using shape as an additional constraint on the similarity search. The ligand's surface or a set of user-defined spheres may be used as an inclusive constraint, the receptor's surface or a set of user-defined spheres may be used as an exclusive constraint, or both inclusive and exclusive shapes may be used. Multiple spheres may be defined for both inclusive and exclusive search by choosing the "Spheres" option in the dropdown menu and pressing "Add". This will display a menu allowing the user to set the position and size of the sphere. Pressing "Add" again will enable the user to continue adding spheres. The shape constraint is applied differently depending on whether it is used to search the database prior to pharmacophore filtering or to filter results returned by a pharmacophore search. In the former case, a shape search returns molecules in the search database that contain the entire included shape and do not overlap any part of the excluded shape. In the latter case, the results of the pharmacophore search are filtered by shape to ensure that at least one heavy atom falls within the inclusive shape and no heavy atom centers fall within any exclusive shape. In either case, the results of the initial search are aligned to that search for the secondary filtering - thus for a given molecule a pose may exist that allows it to meet the secondary filter, but it will not be returned if the aligned pose resulting from the initial search does not meet those constraints. Our recommendation is that users define both inclusive and exclusive shape constraints using the ligand and receptor surfaces and use them as secondary filtering on the results of a pharmacophore search. The order of application of pharmacophore and shape constraints can be swapped using the second button from the top of the left sidebar, which switches between "Pharmacophore Search → Shape Filter" and "Shape Search → Pharmacophore Filter".
