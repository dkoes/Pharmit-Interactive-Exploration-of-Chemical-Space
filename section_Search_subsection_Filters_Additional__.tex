\section{Search}

search modality - differences

\subsection{Filters}
Additional hit reduction and screening options are provided in the "Filters" menu. The three options for hit reduction include restricting the maximum hits returned for every configuration, for every molecule, and overall. The maximum hits per configuration constraint is applied greedily, returning the first n results found without prioritizing any other aspects of those results. The other two constraints are applied after the results are sorted, giving priority to those of higher rank (lower RMSD for pharmacophore searches and higher similarity scores for shape searches). There are seven options given for hit screening. These include constraints on molecular weight, the number of rotatable bonds, LogP (a measure of lipophilicity), polar surface area (indicative of ability to permeate cell membranes), the number of aromatic groups, the number of hydrogen bond acceptors, and the number of hydrogen bond donors. Properties are computing using OpenBabel. 

\subsection{Hit Reduction}

\subsection{Result Browser}
When a database search is initiated, a right sidebar opens to display the search results. This sidebar may be hidden by pressing the right-facing arrow at the top left of the sidebar, or it may be closed by pressing the "x" in the upper right corner. Results may be sorted in increasing or decreasing order based on RMSD, mass, or number of rotatable bonds; the sorting method is determined by pressing on the name of desired sorting method, and pressing the name again to toggle between increasing and decreasing order. Lavender arrowheads to the right of the sorting method names indicate whether sorting is being performed in increasing (up) or decreasing (down) order. 