\section{Search}

search modality - differences
\cite{matchpack}
\cite{Koes_2011}
 The shape constraint is applied differently depending on whether it is used to search the database prior to pharmacophore filtering or to filter results returned by a pharmacophore search. In the former case, a shape search returns molecules in the search database that contain the entire included shape and do not overlap any part of the excluded shape. In the latter case, the results of the pharmacophore search are filtered by shape to ensure that at least one heavy atom falls within the inclusive shape and no heavy atom centers fall within any exclusive shape. In either case, the results of the initial search are aligned to that search for the secondary filtering - thus for a given molecule a pose may exist that allows it to meet the secondary filter, but it will not be returned if the aligned pose resulting from the initial search does not meet those constraints. Our recommendation is that users define both inclusive and exclusive shape constraints using the ligand and receptor surfaces and use them as secondary filtering on the results of a pharmacophore search. The order of application of pharmacophore and shape constraints can be swapped using the second button from the top of the left sidebar, which switches between "Pharmacophore Search → Shape Filter" and "Shape Search → Pharmacophore Filter".
 
\subsection{Filters}
Additional hit reduction and screening options are provided in the "Filters" menu. The three options for hit reduction include restricting the maximum hits returned for every configuration, for every molecule, and overall. The maximum hits per configuration constraint is applied greedily, returning the first n results found without prioritizing any other aspects of those results. The other two constraints are applied after the results are sorted, giving priority to those of higher rank (lower RMSD for pharmacophore searches and higher similarity scores for shape searches). There are seven options given for hit screening. These include constraints on molecular weight, the number of rotatable bonds, LogP (a measure of lipophilicity), polar surface area (indicative of ability to permeate cell membranes), the number of aromatic groups, the number of hydrogen bond acceptors, and the number of hydrogen bond donors. Properties are computing using OpenBabel. 

\subsection{Hit Reduction}

\subsection{Result Browser}
When a database search is initiated, a right sidebar opens to display the search results. This sidebar may be hidden by pressing the right-facing arrow at the top left of the sidebar, or it may be closed by pressing the "x" in the upper right corner. Results may be sorted in increasing or decreasing order based on RMSD, mass, or number of rotatable bonds; the sorting method is determined by pressing on the name of desired sorting method, and pressing the name again to toggle between increasing and decreasing order. Lavender arrowheads to the right of the sorting method names indicate whether sorting is being performed in increasing (up) or decreasing (down) order. 