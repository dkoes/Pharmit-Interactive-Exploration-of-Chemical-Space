\section{Introduction}

There are a multitude of software packages and web services that assist in computer aided drug design \cite{Villoutreix_2013}, but a relative paucity of web services that support structure-based virtual screening.  Those that exist, such as DockBlaster \cite{Irwin_2009}, iDrug \cite{Wang_2014}, iStar \cite{Li_2014}, e-LEA3D \cite{Douguet_2010}, and MTiOpenScreen \cite{Labb__2015}, are typically batch-processing services where the user submits a virtual screening job and receives the results hours or days later. They are also usually limited to screening a pre-determined library of compounds of limited size. Alternatively, advanced algorithms enable interactive time-scale searches, but existing web resources \cite{Koes_2012,Koes_2012z} are limited by a single search modality and a restricted search domain.  In contrast, Pharmit provides both pharmacophore and molecular shape search modalities as well as ranking of results by energy minimization, and, in addition to providing a variety of pre-built compound libraries, allows users to upload their own compound libraries for screening. 

Pharmit takes as its input structure files of a receptor and/or ligand. Structures may be provided by the user or extracted directly from the Protein Data Bank (PDB). 
Pharmacophore and/or molecular shape queries are created and edited in a modern interactive interface powered by 3Dmol.js \cite{Rego_2014}, which provides high performance 3D molecular graphics without the need for plugins or Java. Once a query is defined, the user selects and searches a compound library for matching compounds.  Results are typically returned in seconds and are displayed in-browser.  A variety of filtering and ranking criteria can be applied, and hits can be further refined and ranked using energy minimization. Structure files of the query-optimized hit compounds can be downloaded, and the full session state can be saved and restored.  In total, Pharmit provides a comprehensive online platform for structure-based virtual screening.





