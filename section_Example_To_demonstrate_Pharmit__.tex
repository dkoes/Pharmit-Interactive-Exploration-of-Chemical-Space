\section{Example}

To demonstrate Pharmit's capabilities and typical usage, this section describes a virtual screen of Tyrosin-protein kinase C-SRC based off a complex (PDB 2SRC) with a nonhydrolyzable ATP analog (ANP). The resulting query is available as an interactive example from the Pharmit examples page.  For this example we submitted the compounds of the the DUDe \cite{Mysinger_2012} SRC target benchmark as a contributed library.  Compounds were renamed to include the keyword `active' if they were active compounds, which allows Pharmit to automatically compute the enrichment factor (EF) and F1 score (geometric mean of recall and precision) of a search.

  Beginning on the Pharmit main page, the user initiates a search by typing `2SRC' into the `start from PDB' box. This will retrieve the ligand names, ANP and PTR, from the PDB file, and they are displayed in the dropdown menu next to the PDB code box. The user then selects the ANP ligand and, for this example, chooses to ignore the binding site waters. After clicking `submit,' the user will be taken to the main Pharmit interface. A set of interacting pharmacophore features will be automatically generated from the protein-ligand complex.  The user then selects the DUDe SRC Benchmark from the Contributed Libraries list in the search selection menu.  This contains 514,797 conformers of 35,024 molecules, of which 524 are known actives.
  
  Pharmit identifies 26 pharmacophore features in ANP, 14 of which are interacting with the receptor.  Searching with this default query yields no hits as it is overly specific.  In general, we find that queries rarely benefit from having more than 5 distinct features.
  To generate a useful query for 2SRC, we rationally target the canonical hinge site, with which ANP strongly interacts, by including a hydrogen donor and acceptor and an aromatic ring to mimic the adenine moiety.  Searching with this reduced query produces 92,398 hits (matching conformations) with an EF of 2.8.  Adding a directionality constraint to the aromatic ring reduces the number of hits by 2,000 and slightly increases the EF.  The user can then interactively explore adding additional interacting features to the query.  The addition of a hydrogen acceptor interaction on the ribose of the ligand, as shown in Figure~\ref{}, generates 1,881 hits with an EF of 16.8.  The addition of an exclusive shape constraint with a tolerance of 2.5, which filters out compounds with severe steric clashes with the receptor, decreases the number of hits to 1,248 and increases the EF to 18.4. Alternative queries can yield even higher enrichment factors at the expense of lower F1 scores due to fewer total active compounds being returned.  In this case the user is guided by a benchmark library, but a similar interactive exploration is possible through rational investigation of the hit compounds of query.
Query results can be minimized to produce a more meaningful ranking, as shown in Figure~\ref{}, where the top 13 compounds, and 16 of the top 20, are correctly identified actives. 
  
